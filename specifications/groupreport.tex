\documentclass[12pt,a4paper,twoside]{article}

\setlength{\oddsidemargin}{-0.4mm} % 25 mm left margin - 1 in
\setlength{\evensidemargin}{\oddsidemargin}
\setlength{\topmargin}{-5.4mm} % 20 mm top margin - 1 in
\setlength{\textwidth}{160mm} % 25 mm right margin
\setlength{\textheight}{247mm} % 20 mm bottom margin
\setlength{\headheight}{5mm}
\setlength{\headsep}{5mm}
\setlength{\parindent}{0mm}
\setlength{\parskip}{\medskipamount}
\usepackage{graphicx}
\newcommand{\subsubsubsection}[1]{\vspace{0.01in}\textbf{#1}\vspace{0.01in}\\}
\usepackage{float}
\usepackage{array}
\raggedbottom


\begin{document}

\begin{titlepage}
\begin{center}

\textsc{\LARGE University of Cambridge}\\[3.5cm]

\textsc{\Large Computer Science Tripos \\[2mm] Part 1B Group Project 2013-2014}\\[0.4cm]
\textsc{\Large Team Foxtrot - Money World}\\[2cm]

{\huge \bfseries \vspace{3.5mm} Group Report}\\[2cm]

\begin{center}
\large
Daniel Low\\
Darren Foong\\
Jovan Powar\\
Samuel Haines\\
\end{center}

\vfill

{\large Last Revised: \today}
\end{center}
\end{titlepage}

\newpage
\thispagestyle{empty}
\cleardoublepage
\newpage

\section{Summary}

At the conclusion of this project, we have produced an Android application which meets most of the requirements as specified in the specification.

The user can view visualisations and also regional view of indicators. The visualisations are displayed in a manner that is from the least detailed view (summary) to the most detailed view. At the start of the application, the user will see a summary visualisation of an indicator and can swipe the screen to change indicators. More details about the particular indicator can be found by tapping the summary visualisation, which will bring the user into another screen with more details. Explanations of an indicator is given by tapping the information icon. If the user wants to see how countries in the region compare for that particular indicator, the user can tap the map button which will bring up a map view that visualises the indicator by filling the country region with a colour from the range of red, amber and green.

\section{Successes}

\begin{itemize}
	\item We managed to create an Android application that shows relevant data about a user’s country and obtains said data from a custom server backend.
	\item We managed to create a server backend that provided a uniform RESTful interface for clients to retrieve data from.
	\item The user interface followed the specifications to a large extent and is easy to use.
	\item We managed to keep the code as modular as possible so that people could work on different components without worrying about complications. This was made easier by the MVC framework of Sencha.
	\item We created easy and simple visualisations and the easy user interface allows the user to gain access to more data if he/she wishes so.
\end{itemize}

\section{Failures}

\begin{itemize}
	\item Some features mentioned in the original specifications were not implemented, for example, aggregation from various sources on the backend, and comments on the frontend. This was due to the shortage of manpower and the loss of time in climbing the steep learning curve for frameworks and technologies that members may not have used before.
	\item In the beginning, we did not fully understand the client’s requirements for visualisations, which we only found out and clarified later (and focused strongly on). This resulted in some time wasted which could have been used better.
	\item As stated, time was lost in learning frameworks that were themselves imperfect; many times we found ourselves having to devise an inelegant hack to solve a problem.
\end{itemize}

\section{Lessons learnt}

When we first started this project, we overestimated the ease of use of several available libraries out there. For instance, when we decided to settle for a web framework that would be used to generate our application framework and visualisation, we were faced with a few challenges. The first was to familiarise ourselves with the framework which was not trivial because API for the framework is different from what we are used to. Secondly, working with the default implementations of the framework sometimes became too limited and we had to work around the library code to achieve what we wanted. For example, interfacing with the canvas Object generated by the framework was different than that of a normal HTML canvas. The lesson learnt here is to be familiar with the the tools we know. Because none of us used Sencha before, perhaps spending a week at the start going through tutorials as a group would have been a useful activity.

On a similar note, another lesson we learned is to have share similar workflow so that everyone has similar working environment. Because we each worked in our separate module before using the framework, we had different working environment. Some of us were using local web server and some were using the build tools from Sencha. As a result, there were several problems when trying to debug a problem that another member had because the layout of the folder could be different. It is worth investing time in the beginning to a common work style to prevent time wastage in solving such problems later on.

We found that several requirements that was set out in the specifications were later altered or completely dropped because of the irrelevance or the technical limitations of our available technology. While specifications should not change after it has been decided, we learned that for projects that involves user interaction, the requirements are quite hard to pin down without creating prototypes and testing it on people.

During our journey, we saw many hiccups, such as having one member leave us halfway. That affected us because we had to redistribute our tasks and it made it more challenging for us. This teaches us that we cannot assume project to go as planned and we should come up with contingency plans to deal with misfortune.

\section{Summary of team members' contributions}

\subsection{Daniel Low}

\begin{itemize}
	\item Design phase
	\begin{enumerate}
		\item Wireframing
		\item Specification
	\end{enumerate}
	\item Implementation phase
	\begin{enumerate}
		\item Map module
		\item Integrating canvas and some javascript graphing libraries with sencha
		\item Implemented some charts and graphics visualisation
		\item General debugging and testing of the whole application
	\end{enumerate}
\end{itemize}

As the project manager, I was responsible for a variety of non-technical tasks on top of my contributions as a developer. Before each client meeting, I was responsible for consolidating information from all team members and producing a report to be presented to the client. For the specification, I was in charge of coming up with application wireframing. 

Additionally, I actively kept track of the progress of members and also organise meetings to discuss our ideas and progress. I set up the collaborative working environment such as google docs, and a facebook group to facilitate group discussion. I also helped set up the work environment on members computer when they had problems.

In my role as a developer, I have focused on the map module. I implemented a map module to provide a regional view. Furthermore, I also implemented glue code for external libraries to be used in the framework that we were using, allowing other web features such as using the canvas object and other plotting libraries such as Rgraph to work. Towards the end, due to the lack of manpower, I shifted to work on visualisation module and came up with some charts and canvas visualisations.

In the last leg of the development phase, I switched to focusing on the UI and UX component of the application. This involves testing, fixing bugs, and styling the main application so that it looks reasonably similar to a native Android application.

\subsection{Darren Foong}

\begin{itemize}
	\item Design phase
	\begin{enumerate}
		\item Specification
		\item Database design
	\end{enumerate}
	\item Implementation phase
	\begin{enumerate}
		\item Server backend
		\item Client application framework
		\item Visualisations
	\end{enumerate}
\end{itemize}

During the design phase, I worked with Samuel to design the database schema, functionality, and specifications of the backend. I researched about the appropriate tools and technologies required to build the backend for this project and designed the interface with which the client will interact with the server. I also researched about the various sources of data that we could potentially use for the application and determined the feasibilities of using each.

During the implementation phase, I first wrote the server backend in Java using the Jersey RESTful Web Services framework with a MySQL database. I hosted the application on Google App Engine and created a convenient URL based interface with which the client can use to obtain data in JSON easily from the database.

Due to the shortage of manpower, I proceeded to work on the client. I established the structure for the client using the Sencha Touch framework and put together the various views to create a prototype of the client with a barebones user interface. I designed a basic programming interface for the client for the other group members to be able to add views and extensions to the client easily. I constantly refined the structure and organisation of the various modules of the client, helping my other group members to integrate their modules into Sencha. I also refined the user interface in terms of the menus and navigation through the client.

Finally, I put together visualisations and charts using basic technologies and Sencha’s built-in libraries. I wrote client code to request the server for JSON data, which was used to provide raw data for the visualisations.

\subsection{Samuel Haines}

\begin{itemize}
	\item Design phase
	\begin{enumerate}
		\item Specification
		\item Data source research
		\item Database design
	\end{enumerate}
	\item Implementation phase
	\begin{enumerate}
		\item Database population
		\item Election data gathering
		\item Visualisations
	\end{enumerate}
\end{itemize}

During the design phase, I worked with the other group members to produce a specification based on the design brief. From this, I researched several potential sources of data for the application, before suggesting the World Bank due to the ease of use and large amount of data available for a wide variety of countries. Darren and I designed the database that would hold the information.

I wrote a program in Java that downloads all of the available information for all countries from the World Bank’s website and populates the MySQL tables for later retrieval.
I collated data on the dates of elections for the countries in the two selected regions, so that we could display these on the date-slider. However, due to time restrictions we were unable to implement this in the application and instead settled for a library slider implementation.

One visualisation that I worked on was a treemap visual, where different sized boxes represent the different values of a given statistic for countries in a given region. This was not included in the application since it did not fit in well with the style of other visualisations in the application. Another visualisation I created was the life expectancy infographic, where the figure is written on a birthday cake.

\subsection{Jovan Powar}

\begin{itemize}
	\item Design phase
	\begin{enumerate}
		\item Wireframing
		\item Specification
		\item UI structure
		\item Visualisations
	\end{enumerate}
	\item Implementation phase
	\begin{enumerate}
		\item Bubble cloud module (scrapped due to time and personnel constraints)
		\item Implementing data visualisations and text summaries
	\end{enumerate}
\end{itemize}

During the design phase, I was in charge of producing a user interface design and visualisations. I was involved in deciding the depth of presentation of economic data and with wireframing. After the initial client meeting I led an additional phase of UI design, and designed additional visualisations.

During implementation, I was in charge of the bubble cloud module. I arrived at an implementation that produced a visualisation for certain data sets, with plans to integrate it with the app’s data and interface. However due to lack of manpower this module was dropped and I shifted to work on visualisation modules, producing the text-based, circular, and income inequality views.

\end{document}
